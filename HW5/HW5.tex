\documentclass{article}

\usepackage[utf8]{inputenc}
\usepackage{fancyhdr}
\usepackage{extramarks}
\usepackage{amsmath}
\usepackage{amsthm}
\usepackage{amsfonts}
\usepackage{tikz}
\usepackage[plain]{algorithm}
\usepackage{algpseudocode}
\usepackage{listings}

\usetikzlibrary{automata,positioning}

%
% Basic Document Settings
%

\topmargin=-0.45in
\evensidemargin=0in
\oddsidemargin=0in
\textwidth=6.5in
\textheight=9.0in
\headsep=0.25in

\linespread{1.1}

\pagestyle{fancy}
\lhead{\hmwkAuthorName}
%\chead{\hmwkClass\ (\hmwkClassInstructor\ \hmwkClassTime): \hmwkTitle}
\rhead{\firstxmark}
\lfoot{\lastxmark}
\cfoot{\thepage}

\renewcommand\headrulewidth{0.4pt}
\renewcommand\footrulewidth{0.4pt}

\setlength\parindent{0pt}

%
% Create Problem Sections
%

\newcommand{\enterProblemHeader}[1]{
    \nobreak\extramarks{}{Problem \arabic{#1} continued on next page\ldots}\nobreak{}
    \nobreak\extramarks{Problem \arabic{#1} (continued)}{Problem \arabic{#1} continued on next page\ldots}\nobreak{}
}

\newcommand{\exitProblemHeader}[1]{
    \nobreak\extramarks{Problem \arabic{#1} (continued)}{Problem \arabic{#1} continued on next page\ldots}\nobreak{}
    \stepcounter{#1}
    \nobreak\extramarks{Problem \arabic{#1}}{}\nobreak{}
}

\setcounter{secnumdepth}{0}
\newcounter{partCounter}
\newcounter{homeworkProblemCounter}
\setcounter{homeworkProblemCounter}{1}
\nobreak\extramarks{Problem \arabic{homeworkProblemCounter}}{}\nobreak{}

%
% Homework Problem Environment
%
% This environment takes an optional argument. When given, it will adjust the
% problem counter. This is useful for when the problems given for your
% assignment aren't sequential. See the last 3 problems of this template for an
% example.
%
\newenvironment{homeworkProblem}[1][-1]{
    \ifnum#1>0
        \setcounter{homeworkProblemCounter}{#1}
    \fi
    \section{Problem \arabic{homeworkProblemCounter}}
    \setcounter{partCounter}{1}
    \enterProblemHeader{homeworkProblemCounter}
}{
    \exitProblemHeader{homeworkProblemCounter}
}

%
% Homework Details
%   - Title
%   - Due date
%   - Class
%   - Section/Time
%   - Instructor
%   - Author
%

\newcommand{\hmwkTitle}{Homework\ \#5}
\newcommand{\hmwkDueDate}{April 11, 2019}
\newcommand{\hmwkClass}{Programming Languages}
\newcommand{\hmwkClassTime}{Section 101}
\newcommand{\hmwkClassInstructor}{Erin Keith}
\newcommand{\hmwkAuthorName}{\textbf{Michael DesRoches}}

%
% Title Page
%

\title{
    \vspace{2in}
    \textmd{\textbf{\hmwkClass:\ \hmwkTitle}}\\
    \normalsize\vspace{0.1in}\small{Due\ on\ \hmwkDueDate\ at 9:00am}\\
    \vspace{0.1in}\large{\textit{\hmwkClassInstructor\ \hmwkClassTime}}
    \vspace{3in}
}

\author{\hmwkAuthorName}
\date{}

\renewcommand{\part}[1]{\textbf{\large Part \Alph{partCounter}}\stepcounter{partCounter}\\}


\begin{document}
\maketitle
\pagebreak

\begin{homeworkProblem}

\begin{lstlisting}
(25pts) Suppose  we  are  compiling  for  a  machine  with  1-byte  characters,
2-byte  shorts,  4-byte  integers,  and  8-byte  reals,  and  with  alignment
rules  that  require  the  address  of  every primitive data element to be an
even multiple of the element’s size. Suppose further that the compiler  is
not  permitted  to  reorder  fields.  How  much  space  will  be  consumed  by
the following array? Explain.

  A : array [0..9] of record
                            s : short
                            c : char
                            t : short
                            d : char
                            r : real
                            i : integer
\end{lstlisting}

  \textbf{Solution}
  \\ \\
  A total of space consumed will be 240 bytes. \\
  s has an offset of 0.\\
  c has an offset of 2 \\
  t holds an offset of 4, compelled by 1 (means of allignment)\\
  d holds an offset of 6 \\
  r holds an offset of 8 \\
  added up, we get 20 bytes - not a multiple of 8 so each array element is padded to 24 bytes\\
  when adding, "i," which holds an offset of 16, \\
  The array has 10 elements, 24 times 10 is equal to 240 bytes\\



\end{homeworkProblem}
\pagebreak

\begin{homeworkProblem}

\begin{lstlisting}
(25pts) For  the  following  code  specify  which  of  the  variables  a,b,c,d
are  type  equivalent under (a) structural equivalence, (b) strict name
equivalence, and (c) loose name equivalence.

Type  T = array [1..10] of integer
      S = T
a: T
b: T
c: S
d: array [1..10] of integer
\end{lstlisting}

  \textbf{Solution}\\
  (a) All have structural equivalence\\
  (b) Variables a and b have strict name equivalence\\
  (c) Variables a, b, and c have loose name equivalence\\

\end{homeworkProblem}
\pagebreak

\begin{homeworkProblem}
\begin{lstlisting}
(25pts) We are trying to run the following C program:

typedef struct
{
  int a;
  char *b;
} Cell;

void AllocateCell(Cell* q)
{
  q= (Cell*) malloc ( sizeof(Cell) );
}

void main ()
{
  Cell* c;
  AllocateCell(c);
  c->a= 1;free(c);
}

The program produces a run-time error. Why? Rewrite the functions AllocateCell
and mainso that the program runs correctly.

\end{lstlisting}

  \textbf{Solution}
    \begin{lstlisting}
    typedef struct
    {
      int a ;
      char * b ;
    } Cell ;

    void AllocateCell ( Cell ** q )
    {
      * q = ( Cell *) malloc ( sizeof ( Cell ));
    }
    void main ()
    {
      Cell * c ;
      AllocateCell (& c ); // if you pass c then it is passed by value ,
                          // we use &c because we want to pass by reference
      c - > a =1;
      free ( c );
    }
    \end{lstlisting}
\end{homeworkProblem}
\pagebreak

\begin{homeworkProblem}
\begin{lstlisting}
(25 pts) Consider the following C declaration, compiled on a 32-bit Pentium
machine (with array elements aligned at addresses multiple of 4 bytes).

struct
{
  int n;
  char c;
} A[10][10];

If  the  address  of A[0][0]is  1000  (decimal),  what  is  the  address
of A[3][7]?Explainhow this is computed.

\end{lstlisting}

  \textbf{Solution}\\
  \begin{lstlisting}
  struct
  {
  int n ;
  char c ;
  } A [10] [10]; // row , colomn total size of 10
  
  The address of A[3][7] is 1296 So the size of the structure is 8. The base a
  ddress of A[0][0] is 1000. Now we do the computations:
  (1000) + (3 ∗ 10 ∗ 8) + (7 ∗ 8) = 1296

  COMPUTATION:
  (base address) + (element row 3 * total size of rows * size of struct) +
  (element column 7 * size of struct)
  \end{lstlisting}
\end{homeworkProblem}
\pagebreak

\begin{homeworkProblem}
\begin{lstlisting}

(Extra Credit -10 pts) Write a small fragment of code that shows how unions
can be used in C to interpret the bits of a value of one type as if they
represented a value of some other type (non-converting type cast).

\end{lstlisting}

  \textbf{Solution}\\
  \begin{lstlisting}

union ch_ascii
{
    char ch ;
    unsigned int asc_val ;
} union_tag ; // union variable

void main ()
{
    union_tag . ch = ’C’;
    union_tag . asc_val = (int )( union_var . ch ); // non converting type
                                                    //cast occuring here
    // output of program C and 67
    printf ("%c␣%d", union_tag . ch , union_tag . asc_val );
    return 0;
}
\end{lstlisting}


\end{homeworkProblem}
\pagebreak

\end{document}
