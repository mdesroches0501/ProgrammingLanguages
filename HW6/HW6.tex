\documentclass{article}

\usepackage[utf8]{inputenc}
\usepackage{fancyhdr}
\usepackage{extramarks}
\usepackage{amsmath}
\usepackage{amsthm}
\usepackage{amsfonts}
\usepackage{tikz}
\usepackage[plain]{algorithm}
\usepackage{algpseudocode}
\usepackage{listings}

\usetikzlibrary{automata,positioning}

%
% Basic Document Settings
%

\topmargin=-0.45in
\evensidemargin=0in
\oddsidemargin=0in
\textwidth=6.5in
\textheight=9.0in
\headsep=0.25in

\linespread{1.1}

\pagestyle{fancy}
\lhead{\hmwkAuthorName}
%\chead{\hmwkClass\ (\hmwkClassInstructor\ \hmwkClassTime): \hmwkTitle}
\rhead{\firstxmark}
\lfoot{\lastxmark}
\cfoot{\thepage}

\renewcommand\headrulewidth{0.4pt}
\renewcommand\footrulewidth{0.4pt}

\setlength\parindent{0pt}

%
% Create Problem Sections
%

\newcommand{\enterProblemHeader}[1]{
    \nobreak\extramarks{}{Problem \arabic{#1} continued on next page\ldots}\nobreak{}
    \nobreak\extramarks{Problem \arabic{#1} (continued)}{Problem \arabic{#1} continued on next page\ldots}\nobreak{}
}

\newcommand{\exitProblemHeader}[1]{
    \nobreak\extramarks{Problem \arabic{#1} (continued)}{Problem \arabic{#1} continued on next page\ldots}\nobreak{}
    \stepcounter{#1}
    \nobreak\extramarks{Problem \arabic{#1}}{}\nobreak{}
}

\setcounter{secnumdepth}{0}
\newcounter{partCounter}
\newcounter{homeworkProblemCounter}
\setcounter{homeworkProblemCounter}{1}
\nobreak\extramarks{Problem \arabic{homeworkProblemCounter}}{}\nobreak{}

%
% Homework Problem Environment
%
% This environment takes an optional argument. When given, it will adjust the
% problem counter. This is useful for when the problems given for your
% assignment aren't sequential. See the last 3 problems of this template for an
% example.
%
\newenvironment{homeworkProblem}[1][-1]{
    \ifnum#1>0
        \setcounter{homeworkProblemCounter}{#1}
    \fi
    \section{Problem \arabic{homeworkProblemCounter}}
    \setcounter{partCounter}{1}
    \enterProblemHeader{homeworkProblemCounter}
}{
    \exitProblemHeader{homeworkProblemCounter}
}

%
% Homework Details
%   - Title
%   - Due date
%   - Class
%   - Section/Time
%   - Instructor
%   - Author
%

\newcommand{\hmwkTitle}{Homework\ \#6}
\newcommand{\hmwkDueDate}{April 25, 2019}
\newcommand{\hmwkClass}{Programming Languages}
\newcommand{\hmwkClassTime}{Section 101}
\newcommand{\hmwkClassInstructor}{Erin Keith}
\newcommand{\hmwkAuthorName}{\textbf{Michael DesRoches}}

%
% Title Page
%

\title{
    \vspace{2in}
    \textmd{\textbf{\hmwkClass:\ \hmwkTitle}}\\
    \normalsize\vspace{0.1in}\small{Due\ on\ \hmwkDueDate\ at 9:00am}\\
    \vspace{0.1in}\large{\textit{\hmwkClassInstructor\ \hmwkClassTime}}
    \vspace{3in}
}

\author{\hmwkAuthorName}
\date{}

\renewcommand{\part}[1]{\textbf{\large Part \Alph{partCounter}}\stepcounter{partCounter}\\}


\begin{document}
\maketitle
\pagebreak

\begin{homeworkProblem}

\begin{lstlisting}
(24pts) Consider the following C program:
  void foo()
  {
    int i;
    printf("%d ", i++);
  }
  void main()
  {
    int j;
    for (j = 1;
    j <= 10; j++)
    foo();
  }

Local variable i in subroutine foo is never initialized. On many systems,
however, variable i appears to “remember” its value between the calls to
foo, and the program will print 0 1 2 3 4 5 6 7 8 9.
  (a)(12pts) Suggest an explanation for this behavior.
  (b)(12pts) Change the code above (without  modifying
     function foo) to alter this behavior.

\end{lstlisting}

  \textbf{Solution}
  \\ \\
(a) The local variable, i, is never initialized. Therefore, the value isn't really\\
known. This will cause undefined behavior depending on the compiler. We can't\\
know the outcome and the outcome may be different on different machines.\\

\begin{lstlisting}
(b)
static void overWriteMemoryLocation(void)
{
    int j = 10;
}
void foo()
{
    int i;
    printf("%d ", i++);
}
int main()
{
    int j;
    for (j = 1;
    j <= 10; j++)
    foo();
}
\end{lstlisting}
\\
Calling another function before foo() overwrites the memory location that foo()\\
will use for i.
\end{homeworkProblem}
\pagebreak

\begin{homeworkProblem}

\begin{lstlisting}
(20pts) Can you write a macro in C that “returns” the factorial of aninteger
argument(without calling a subroutine)? Why or why not?
\end{lstlisting}

  \textbf{Solution}\\
No, you can't have macro in C w/out calling a soubroutine. Macro is a \\
preprocessor before the code is compiled. Preprocessors include the headers \\
and text replace the macros with their definition. So, any recursive definition\\
of a macro will be replaced only once and not until the base condition is met. \\

\end{homeworkProblem}

\begin{homeworkProblem}
\begin{lstlisting}
(24pts) Consider a subroutine swap that takes two parameters and simply swaps
their values.For example, after calling swap(X,Y), X should have the original
value of Y and Y the original value of X. Assume that variables to be swapped
can be simple or subscripted (elements  of  an  array), and they have the same
type (integer). Show that it is impossible to write such a general-purpose
swap subroutine in a language with:

  (a)(12pts) parameter passing by value.
  (b)(12pts) parameter passing by name.

Hint: for the case of passing by name, consider mutually dependent parameters.

\end{lstlisting}

  \textbf{Solution}
    \begin{lstlisting}
A subroutine is termed as an executable code resided with in the block or in a
function. Consider the sub routine for the swap program in the language:

(a) Parameter passing by value:

    procedure swap(x, y: int):
              var t : int
                   t:=x;
                   x:=y;
                   y:=t;
    end;

    Main:
         Declare i,j;
         Call swap(i,j)

By calling the parameters through pass by values, none of the actual
arguments can be changed, so the variables retain the values they are
initialized with.

Flow for the function call by its value:

    temp:=i
    i:=j
    j=temp

In this scenario the subroutine will not reflect the function. Hence it
shows is impossible to write the code for the elements of the array.

(b) Parameter passing by name:

    procedure swap(x, y: int):
              var t : int
                   t:=x;
                   x:=y;
                   y:=t;
    end;

    Main:
         Declare i,a[];
         Call swap(i,a)

By calling the subroutine by passed by name conflicts are taken care of
between actual parameters and the local variables of the function.

Flow for the function call by its name:

    temp:=i
    i:=a
    a=temp

Call by name can’t handle the swap properly by calling its name.
    \end{lstlisting}
\end{homeworkProblem}
\pagebreak

\begin{homeworkProblem}
\begin{lstlisting}
(32pts) Consider the following program, written in no particular language. Show
what the program prints in the  case of parameter passing by (a) value,
(b) reference, (c) value-result, and (d) name. Justify your answer. When
analyzing the case of passing by value-result, you may have noticed that there
are two potentiallyambiguousissues–what are they?

procedure f (x, y, z)
  x := x + 1
  y := z
  z := z + 1

// main
  i := 1;
  a[1] := 10;
  a[2] := 11
  f (i, a[i], i);
  print (i);
  print (a[1]);
  print (a[2]);

\end{lstlisting}

  \textbf{Solution}\\
(a) 1, 10, 11\\
(b) 3, 2, 11\\
(c) 3, 11, 3 \\
(d) When initially called, i = 1, a[i] = 10. Then the first stetement changes\\
i to 2. Then a[2] is changed to 2, value is set to z which is also the same as\\
x. Then i is changed to 3. Print = 3, 10, 2. Result = 3, 2, 3
\end{homeworkProblem}

\begin{homeworkProblem}
\begin{lstlisting}
(Extra Credit -10 pts) Does a program run faster when the programmer does not
specify values for the optional parameters in a subroutine call?

\end{lstlisting}

  \textbf{Solution}\\
Performance gain is minimal because the subroutine call but the answer is yes\\
because it takes time to to evaluate the parameters that are being called with\\
the function. \\


\end{homeworkProblem}
\pagebreak

\end{document}
