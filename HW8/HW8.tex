\documentclass{article}

\usepackage[utf8]{inputenc}
\usepackage{fancyhdr}
\usepackage{extramarks}
\usepackage{amsmath}
\usepackage{amsthm}
\usepackage{amsfonts}
\usepackage{tikz}
\usepackage[plain]{algorithm}
\usepackage{algpseudocode}
\usepackage{listings}
\usepackage{graphicx}

\usetikzlibrary{automata,positioning}

%
% Basic Document Settings
%

\topmargin=-0.45in
\evensidemargin=0in
\oddsidemargin=0in
\textwidth=6.5in
\textheight=9.0in
\headsep=0.25in

\linespread{1.1}

\pagestyle{fancy}
\lhead{\hmwkAuthorName}
%\chead{\hmwkClass\ (\hmwkClassInstructor\ \hmwkClassTime): \hmwkTitle}
\rhead{\firstxmark}
\lfoot{\lastxmark}
\cfoot{\thepage}

\renewcommand\headrulewidth{0.4pt}
\renewcommand\footrulewidth{0.4pt}

\setlength\parindent{0pt}

%
% Create Problem Sections
%

\newcommand{\enterProblemHeader}[1]{
    \nobreak\extramarks{}{Problem \arabic{#1} continued on next page\ldots}\nobreak{}
    \nobreak\extramarks{Problem \arabic{#1} (continued)}{Problem \arabic{#1} continued on next page\ldots}\nobreak{}
}

\newcommand{\exitProblemHeader}[1]{
    \nobreak\extramarks{Problem \arabic{#1} (continued)}{Problem \arabic{#1} continued on next page\ldots}\nobreak{}
    \stepcounter{#1}
    \nobreak\extramarks{Problem \arabic{#1}}{}\nobreak{}
}

\setcounter{secnumdepth}{0}
\newcounter{partCounter}
\newcounter{homeworkProblemCounter}
\setcounter{homeworkProblemCounter}{1}
\nobreak\extramarks{Problem \arabic{homeworkProblemCounter}}{}\nobreak{}

%
% Homework Problem Environment
%
% This environment takes an optional argument. When given, it will adjust the
% problem counter. This is useful for when the problems given for your
% assignment aren't sequential. See the last 3 problems of this template for an
% example.
%
\newenvironment{homeworkProblem}[1][-1]{
    \ifnum#1>0
        \setcounter{homeworkProblemCounter}{#1}
    \fi
    \section{Problem \arabic{homeworkProblemCounter}}
    \setcounter{partCounter}{1}
    \enterProblemHeader{homeworkProblemCounter}
}{
    \exitProblemHeader{homeworkProblemCounter}
}

%
% Homework Details
%   - Title
%   - Due date
%   - Class
%   - Section/Time
%   - Instructor
%   - Author
%

\newcommand{\hmwkTitle}{Homework\ \#8}
\newcommand{\hmwkDueDate}{May 7, 2019}
\newcommand{\hmwkClass}{Programming Languages}
\newcommand{\hmwkClassTime}{Section 101}
\newcommand{\hmwkClassInstructor}{Erin Keith}
\newcommand{\hmwkAuthorName}{\textbf{Michael DesRoches}}

%
% Title Page
%

\title{
    \vspace{2in}
    \textmd{\textbf{\hmwkClass:\ \hmwkTitle}}\\
    \normalsize\vspace{0.1in}\small{Due\ on\ \hmwkDueDate\ at 9:00am}\\
    \vspace{0.1in}\large{\textit{\hmwkClassInstructor\ \hmwkClassTime}}
    \vspace{3in}
}

\author{\hmwkAuthorName}
\date{}

\renewcommand{\part}[1]{\textbf{\large Part \Alph{partCounter}}\stepcounter{partCounter}\\}


\begin{document}
\maketitle
\pagebreak

\begin{homeworkProblem}
\begin{lstlisting}

\end{lstlisting}
  \textbf{Solution}
\begin{lstlisting}

\end{lstlisting}
\end{homeworkProblem}
\pagebreak

\begin{homeworkProblem}
\begin{lstlisting}
(14pts) Write the rules for a predicate tally(E,L,N), which succeeds if N is the
number of occurrences of element E in list L. The following query shows an
example of using this predicate:

?-tally(3, [1,2,3,1,2,3],N).
N = 2
\end{lstlisting}
  \textbf{Solution}\\
\begin{lstlisting}
tally(_,[],0).
tally(E,[E|L],N) :- tally(E,L,M), N is M+1.
tally(E,[F|L],N) :- not(E=F), tally(E,L,N).
\end{lstlisting}
\end{homeworkProblem}
\pagebreak

\begin{homeworkProblem}
\begin{lstlisting}
(15pts) Define predicates and/2, or/2, nand/2, nor/2, xor/2, and equ/2 (for
logical equivalence) which succeed or fail according to the result of their
respective operations; e.g.and(A,B) will succeed, if and only if both A and B
succeed. Note that A and B can be Prolog goals (not only the constants true
and fail).

?-and(true, true).
true
\end{lstlisting}
  \textbf{Solution}
\begin{lstlisting}
and(A,B) :- call(A), call(B).

or(A,_) :- A, !.
or(_,B) :- B.

nand(A,B) :- not(and(A,B)).
nor(A,B) :- not(or(A,B)).
equ(A,B) :- or(and(A,B), and(not(A),not(B))).
xor(A,B) :- not(equ(A,B)).
\end{lstlisting}
\end{homeworkProblem}
\pagebreak

\begin{homeworkProblem}
\begin{lstlisting}
(15pts) Write the rules for a predicate gcd(X,Y,G), whichdeterminesthe greatest
common divisor of two positive integer numbers. Use Euclid's algorithm:
https://www.khanacademy.org/computing/computer-science/cryptography/
modarithmetic/a/the-euclidean-algorithm?-gcd

(36, 63, G).
G = 9
\end{lstlisting}
  \textbf{Solution}\\
\begin{lstlisting}
gcd(0,X,X) :- X > 0, !.
gcd(X,Y,G) :- X >= Y, X1 is X-Y, gcd(X1,Y,G).
gcd(X,Y,G) :- X < Y, X1 is Y-X, gcd(X1,X,G).
\end{lstlisting}

\end{homeworkProblem}
\pagebreak

\begin{homeworkProblem}
\begin{lstlisting}
(Extra Credit -10 pts) Write the rules for a predicate flatten(A,B), which
succeeds if A is a list (possibly containing sublists), and B is a list
containing all elements in Aand its sublists, but all at the same level. The
following query shows an example of using this predicate:

?-flatten([1, [2, [3, 4]], 5], L).
L = [1, 2, 3, 4, 5]

\end{lstlisting}
  \textbf{Solution}\\
\begin{lstlisting}
flatten(A,B) :- flatten(A,[],B).
flatten(Var, T, [Var|T]) :- var(Var), !.
flatten([],T,T) :- !.
flatten([H|T],TailList,List) :- !, flatten(H,FlatTail,List), flatten(T,TailList,FlatTail).
flatten(NonList,T,[NonList|T]).
\end{lstlisting}
\end{homeworkProblem}
\pagebreak

\end{document}
