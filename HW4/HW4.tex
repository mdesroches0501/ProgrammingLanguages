\documentclass{article}

\usepackage[utf8]{inputenc}
\usepackage{fancyhdr}
\usepackage{extramarks}
\usepackage{amsmath}
\usepackage{amsthm}
\usepackage{amsfonts}
\usepackage{tikz}
\usepackage[plain]{algorithm}
\usepackage{algpseudocode}
\usepackage{listings}

\usetikzlibrary{automata,positioning}

%
% Basic Document Settings
%

\topmargin=-0.45in
\evensidemargin=0in
\oddsidemargin=0in
\textwidth=6.5in
\textheight=9.0in
\headsep=0.25in

\linespread{1.1}

\pagestyle{fancy}
\lhead{\hmwkAuthorName}
%\chead{\hmwkClass\ (\hmwkClassInstructor\ \hmwkClassTime): \hmwkTitle}
\rhead{\firstxmark}
\lfoot{\lastxmark}
\cfoot{\thepage}

\renewcommand\headrulewidth{0.4pt}
\renewcommand\footrulewidth{0.4pt}

\setlength\parindent{0pt}

%
% Create Problem Sections
%

\newcommand{\enterProblemHeader}[1]{
    \nobreak\extramarks{}{Problem \arabic{#1} continued on next page\ldots}\nobreak{}
    \nobreak\extramarks{Problem \arabic{#1} (continued)}{Problem \arabic{#1} continued on next page\ldots}\nobreak{}
}

\newcommand{\exitProblemHeader}[1]{
    \nobreak\extramarks{Problem \arabic{#1} (continued)}{Problem \arabic{#1} continued on next page\ldots}\nobreak{}
    \stepcounter{#1}
    \nobreak\extramarks{Problem \arabic{#1}}{}\nobreak{}
}

\setcounter{secnumdepth}{0}
\newcounter{partCounter}
\newcounter{homeworkProblemCounter}
\setcounter{homeworkProblemCounter}{1}
\nobreak\extramarks{Problem \arabic{homeworkProblemCounter}}{}\nobreak{}

%
% Homework Problem Environment
%
% This environment takes an optional argument. When given, it will adjust the
% problem counter. This is useful for when the problems given for your
% assignment aren't sequential. See the last 3 problems of this template for an
% example.
%
\newenvironment{homeworkProblem}[1][-1]{
    \ifnum#1>0
        \setcounter{homeworkProblemCounter}{#1}
    \fi
    \section{Problem \arabic{homeworkProblemCounter}}
    \setcounter{partCounter}{1}
    \enterProblemHeader{homeworkProblemCounter}
}{
    \exitProblemHeader{homeworkProblemCounter}
}

%
% Homework Details
%   - Title
%   - Due date
%   - Class
%   - Section/Time
%   - Instructor
%   - Author
%

\newcommand{\hmwkTitle}{Homework\ \#4}
\newcommand{\hmwkDueDate}{March 12, 2019}
\newcommand{\hmwkClass}{Programming Languages}
\newcommand{\hmwkClassTime}{Section 101}
\newcommand{\hmwkClassInstructor}{Erin Keith}
\newcommand{\hmwkAuthorName}{\textbf{Michael DesRoches}}

%
% Title Page
%

\title{
    \vspace{2in}
    \textmd{\textbf{\hmwkClass:\ \hmwkTitle}}\\
    \normalsize\vspace{0.1in}\small{Due\ on\ \hmwkDueDate\ at 9:00am}\\
    \vspace{0.1in}\large{\textit{\hmwkClassInstructor\ \hmwkClassTime}}
    \vspace{3in}
}

\author{\hmwkAuthorName}
\date{}

\renewcommand{\part}[1]{\textbf{\large Part \Alph{partCounter}}\stepcounter{partCounter}\\}


\begin{document}
\maketitle
\pagebreak

\begin{homeworkProblem}

\begin{lstlisting}
(24pts) Translate the following expression into (a)postfix and (b)prefix
notation:

  (b + sqrt(b x b - 4 x a x c))/(2 x a)
\end{lstlisting}

  \textbf{Solution}
  \\ \\
  a)
    \begin{multline*}
      (b*sqrt(bb* 4ac* -) +)(2a*) / \\
    \end{multline*}
  b)
    \begin{multline*}
      /(+ b sqrt(- *bb **4ac))(*2a) \\
    \end{multline*}
\end{homeworkProblem}

\begin{homeworkProblem}

\begin{lstlisting}
(26pts) Some languages (e.g., Algol 68) do not employ short-circuit evaluation
for Boolean expressions. However, in such languages an if ... then ... else
construct (which only evaluates the  arm  that  is  needed) can  be  used  as
an  expression  that  returns  a  value.  Show  how  to  use if...then...elseto
achieve the effect of short-circuit evaluation for A and Band for A or B.
\end{lstlisting}

  \textbf{Solution}\\
  For A and B, short- circuit evaluation can look like:\\

  //return FALSE
  A(FALSE) ANDTH B(FALSE)

  //return FALSE
  A(FALSE) ANDTH B(TRUE)

  //return FALSE
  A(TRUE) ANDTH B(FALSE)

  //return TRUE
  A(TRUE) ANDTH B(TRUE)\\

  For A or B, short circuit evaluation can look like:\\

  //return FALSE
  A(FALSE) OREL B(FALSE)

  //return TRUE
  A(FALSE) OREL B(TRUE)

  //return TRUE
  A(TRUE) OREL B(FALSE)

  //return TRUE
  A(TRUE) OREL B(TRUE)
\end{homeworkProblem}
\pagebreak

\begin{homeworkProblem}
\begin{lstlisting}
(24pts) Consider a midtest loop, here written in C, that processes all lines in
the input until a blank line is found:

for (;;)
{
  line = read_line();
  if (all_blanks(line)) break;
  process_line(line);
}

Show  how  you  might  accomplish  the  same  task in  C using  a (a)
whileand(b) doloop,  if breakinstructionswere not available.
\end{lstlisting}

  \textbf{Solution}
    \begin{lstlisting}
      while loop:

      line = read_line();

      while(!all_blank(line)){
        process_line(line);
        line = readline();
      }

      do loop:

      do{
        line = read_line();

        if(!all_blanks(line))
        process_line(line);
      }

      while(!all_blanks(line));
    \end{lstlisting}
\end{homeworkProblem}
\pagebreak

\begin{homeworkProblem}
\begin{lstlisting}
(26pts)Write  a tail-recursivefunction  in  Scheme  to  compute nfactorial
(n! = 1x2x...xn). You will probably want to define a helper function, as
discussed in the textbook.

\end{lstlisting}

  \textbf{Solution}\\
  Hepler functions make our programs easier to read\\

  (define (hepler x function)\\
  (if 0 ? x)\\
  function\\
  (helper(-n1)(* function x))))\\
  \\
  (define factorial x)\\
  (hepler x1)
\\
\end{homeworkProblem}

\begin{homeworkProblem}
\begin{lstlisting}
(Extra  Credit -10pts) Give  an  example  in  C  in  which  an  in-line
subroutine  may  be significantly faster than a functionally equivalent macro.
Give another example in which the macro  is likely  to  be  faster.  Hint:
think  about  applicative  versus  normal-order  evaluation  of arguments.
\end{lstlisting}

  \textbf{Solution}\\
\#define someFunction(someExpression)\{int x = something * 5 + something - something\} \\
\\
inline int someFunction(someExpression)\{something * 3 + something - something\}

\end{homeworkProblem}
\pagebreak

\end{document}
